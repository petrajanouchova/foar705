\usepath[{/home/petra/context/}]

\environment env_dis
\setupbackgrounds[page][background=WatermarkOverlay]
\setuppagenumbering[location=singlesided]
\setuppagenumber[number=8]

\startcomponent chapter06

\startchapter[title={Epigrafická produkce antické Thrákie napříč
staletími}, marking={Epigrafická produkce antické Thrákie napříč
staletími}, reference={chapter06}]

\section[uvod]{Úvod}

V této kapitole se věnuji podrobné charakteristice nápisů a epigrafické produkci v Thrákii obecně v rámci jednotlivých staletí. Zaměřuji především na celkovou charakteristiku epigrafické produkce, s důrazem na popis a šíření projevů mezikulturních kontaktů a vzájemné ovlivňování epigraficky aktivních komunit. Dále sleduji trendy v publikaci nápisů, rozšíření epigrafického zvyku v prostředí urbánních a venkovských komunit. Nadále mě zajímá složení těchto komunit, společenská hierarchie a šíření kulturních prvků typických pro řecky mluvící komunity v rámci osídlení thráckého pobřeží i thráckého vnitrozemí.

Předmětem analýzy jsou nápisy, které obsahují informaci o dataci, a je
tak možné dobu jejich vzniku zařadit do konkrétního časového bodu, či
intervalu. Dataci nápisů přebírám z dat publikovaných v jednotlivých
korpusech, případně přidávám vlastní interpretaci, a to v případech, kdy
je možné jinak nedatovaný nápis alespoň částečně zařadit do časového
období jako nápis \quotation{římský}.\footnote{Konkrétní způsoby relativního datování jsou
podrobně vysvětleny v Apendixu 1, věnovanému metodologii práce a
technickému řešení.} Při
výběru souboru nápisů pro dané století uplatňuji metodu normalizované
datace, podrobně vysvětlenou v kapitole [chapter04::chapter04], a dále pracuji jen s nápisy
datovanými s přesností do jednoho (koeficient 1), či do dvou po sobě
následujících století (koeficient 0,5). Celkový soubor 2036 nápisů s
koeficientem 1 a 0,5 představuje 89,69 \letterpercent{} všech datovaných
nápisů, což představuje dostatečně velký statistický vzorek nápisů
nesoucí patřičnou míru výpovědní hodnoty pro dané století.

\section[6bc]{6. st. př. n. l.\footnote{Dva potenciálně nejstarší nápisy z území
Thrákie pochází z řeckého města Byzantion, které bylo založeno jako
kolonie řecké Megary v roce 667 př. n. l. Dochované nápisy (IK Byzantion
42; 53a; Vyhledání v Heuristu: IDs:14345,14358) je možné datovat pouze
velice široce: zhruba od poloviny 7. st. př. n. l. až do doby klasické,
respektive hellénistické. Jejich datace je natolik široká, že je nelze
přiřadit k jednomu či dvěma konkrétním stoletím. Protože se však může
jednat o první řecky psané epigrafické památky nalezené na území
Thrákie, považuji nutné je zmínit (ač nespadají do vybraných kategorií s
koeficientem 1 či 0,5). Typologicky se jednalo o označení váhové míry, a
označení délky, v jednom případě s uvedenou příslušností ke komunitě
obyvatel Byzantia. Nejpravděpodobnější interpretace poukazuje na místně
používané standardy, nápisy na předmětech udávající jednotnou délkovou a
váhovou míru používanou v Byzantiu a při obchodování právě s Byzantiem.
Pokud akceptujeme jejich ranou dataci, pak by se mohlo jednat o jeden z
prvních projevů politické a ekonomické autority na území Thrákie, a to v
rámci řecké komunity. Bohužel tyto dva nápisy jsou tak ojedinělé, a
nejistě datované, že mohou stejně dobře pocházet z následujících
století, a tudíž jejich výpovědní hodnota je velmi nízká.}}

\subsection[h.d1jert8m3xqn]{Nápisy datované s přesností na jedno
století \high{\goto{{[}3{]}}[ftnt3]}}

Nápisy ze 6. st. př. n. l. pocházejí výhradně z řeckých měst na pobřeží
Černého, Marmarského a Egejského moře. Jedná se výhradně o krátké texty
funerálního, a tedy i soukromého charakteru. Soudě dle formy a obsahu,
pocházejí nápisy z řeckého kulturního prostředí, jehož tradice
následují.

\useURL[url4][][][]\from[url4]\useURL[url5][][][]\from[url5]
\startnarrower[right]
\setupinterlinespace[line=2.8ex]
{\framed
   [width=\hsize, 
    align={width,nothyphenated,verytolerant,stretch}] % maybe also stretch
{{\it Celkem}:~3

{\it Region měst na pobřeží}:~Abdéra 1, Apollónia Pontská 1, Perinthos
(Hérakleia) 1

{\it Region měst ve vnitrozemí}:~0

{\it Celkový počet individuálních lokalit}:~3

{\it Archeologický kontext nálezu}: neznámý 3

{\it Materiál}:~kámen 3 (mramor 1, místní kámen 1 (poros z Mandry), 1
neznámý);

{\it Dochování nosiče}: 100 \letterpercent{} 1, 75 \letterpercent{} 1, nemožno
určit 1

{\it Objekt}: stéla 3

{\it Dekorace}:~reliéfní dekorace 1 nápis (vyskytující se motiv: stojící osoba
1); architektonická dekorace 1 nápis (vyskytující se motiv: florální
motiv 1, anthemion 1)\high{\goto{{[}4{]}}[ftnt4]}

{\it Typologie nápisu}: soukromé nápisy 3, veřejné 0

{\it Soukromé nápisy}:~funerální 3

{\it Veřejné nápisy}:~0

{\it Délka}:~aritm. průměr 3, medián 3, max. délka 3, min. délka 3

{\it Obsah}:~standardní epigrafické formule 1 (funerální)

{\it Identita}: pouze řecká jména, 2 ženy, 1 muž, krátké texty, celkem 3
osoby, 1 osoba na nápis \endgraf}}
\stopnarrower

\subsection[h.raoz6nd2tzuk]{Nápisy datované s přesností na dvě
století \high{\goto{{[}5{]}}[ftnt5]}}

Nápisy datované do 6. a 5. st. př. n. l. pocházejí rovněž z pobřežních
oblastí Černého a Marmarského moře. Typologicky se taktéž jedná o
soukromé nápisy funerálního charakteru, které pocházely z komunit
řeckého původu, které si udržovaly původní charakter a nedocházelo k
prolínání s thráckou kulturou.

\useURL[url6][][][]\from[url6]\useURL[url7][][][]\from[url7]

\placetable[none]{}
\starttable[|lp(1.00\textwidth)|]
\HL
 C Celkem:~3

Region měst na pobřeží:~Apollónia Pontská 1, Perinthos (Hérakleia) 2

Region měst ve vnitrozemí:~0

Celkový počet individuálních lokalit:~3

Archeologický kontext nálezu: funerální 1, neznámý 2

Materiál: kámen 3 (mramor 1, neznámý 2)

Dochování nosiče: 100 \letterpercent{} 1, nemožno určit 2

Objekt: stéla 3

Dekorace:~reliéfní dekorace 3 nápisy; figurální dekorace 3 nápisy
(vyskytující se motiv: stojící osoba 2, sedící osoba 1, zvíře 1);
architektonická dekorace 1 nápis (vyskytující se motiv: naiskos 1)

Typologie nápisu: soukromé nápisy 3, veřejné 0

Soukromé nápisy: funerální 3

Veřejné nápisy:~0

Délka: aritm. průměr 3,3 řádky, medián 3, max. délka 4, min. délka 3

Obsah: bez hledaných termínů, náhrobní kameny vzpomínající na zemřelého

Identita:~pouze řecká jména, krátké texty, 1 osoba na nápise

 C\AR
\HL
\stoptable

\subsection[h.af7cg6kcyf29]{Diskuze: charakteristika epigrafické
produkce v 6. st. př. n. l.}

Do 6. st. př. n. l. je možné zařadit celkem šest nápisů s přesností
datace na jedno století (tři nápisy), či dvě století (tři nápisy, 6. a
5. st. př. n. l.). Nápisy pocházejí z regionu řeckých kolonií na pobřeží
Egejského, Marmarského a Černého moře; konkrétně z regionu obcí Abdéra,
Apollónia Pontská a Perinthos (pozdější Hérakleia). Místa nálezu se
nacházejí přímo na území daných obcí, a tedy i v bezprostřední blízkosti
mořského pobřeží. Jejich vzájemnou polohu v rámci Thrákie ilustruje Mapa
{[}mapa01{]} v Apendixu {[}appendix02{]}.

Charakteristické datační prvky archaických nápisů se vyskytují na
několika exemplářích: směr použitého písma je v některých případech
bústrofédon, např. IG Bulg 1,2 404 z Apollónie Pontské, či je používána
místní epichórická alfabéta z Thasu/Paru, např. I Aeg Thrace 30 z Abdéry
(Petrova 2015, 9). Objekty nesoucí nápis jsou vytvořeny z kamene
převážně místního původu, jako je poros, a z části z mramoru, jehož
původ není znám, ale dá se předpokládat jeho lokální zdroj. Vizuální
podoba objektů nesoucích nápis se velmi podobá nápisům pocházejícím z
ostatních řecky mluvících komunit té doby (Kurtz, Boardman 1971, 68-90,
121-127; Sourvinou-Inwood 1996, 277-297; Petrova 2015, 9-18). Stély v
Apollónii Pontské mají prostý tvar obdélníku, či v Perinthu mají typický
podlouhlý tvar s palmetovým ukončením, známým v literatuře jako
anthemion. Převládající styl dekorace je reliéf, který je zastoupen ze
dvou třetin. Reliéfní vyobrazení představují motivy typické pro náhrobní
stély z řeckých kolonií té doby: ve většině případů se jedná o
vyobrazení nebožtíka s doplňkovými atributy jako je pes, psací tabulka,
maková hlavice apod. Z kontextu řeckých komunit v oblasti Thrákie se z
6. st. př. n. l. se dochovaly i další náhrobní stély podobného
charakteru, které však nenesou žádný nápis, jako například stéla
zobrazující stojícího muže se psem. Toto vyobrazení, které je obecně
považována za zpodobnění řeckého aristokrata, se do dnešní doby
dochovalo celkem v 15 exemplářích po celém Černomoří (Petrova 2015,
11-18) a bývá datováno na přelom 6. a 5. st. př. n. l. Není zcela jasné,
zda se jedná o černomořskou produkci, či o stély importované z jiných
částí řeckého světa. Avšak jejich výskyt právě v Černomoří nasvědčuje na
jejich původ v rámci místních řeckých komunit iónského původu. Podobně
tomu tak může být i u funerální stély IG Bulg 1,2 405 z Apollónie
Pontské s vyobrazením stojícího muže, která dle textu nápisu patřila
Deinéovi, synovi Anaxandra.

\subsubsection[h.y4mq277w1j8j]{Typologie nápisů}

Dochované nápisy z 6. st. př. n. l. je možné typologicky i obsahově
určit jako soukromé nápisy Všechny dochované nápisy jsou funerální
texty, vytvořené za účelem označení místa pohřbu a na památku zesnulého
v nejbližší komunitě. Krátký rozsah textů poukazuje na účelnost sdělení:
typický text obsahuje jméno zesnulého a určení jeho biologického původu,
typicky udáváním jména rodiče, a dále případně zhotovitele nápisu.
Nápisy v 50 \letterpercent{} případů promlouvají k čtenáři:
personifikovaný náhrobní kámen oznamuje komu přesně patří, a jehož život
připomíná.\high{\goto{{[}6{]}}[ftnt6]}~Obsah těchto typů sdělení měl
význam v nejbližší komunitě, kde každý znal zesnulého či jeho rodinu.

Nápisy svým kontextem i obsahem pochází z řeckých komunit a jejich
charakter poukazuje na udržování tradičních společenských norem i v
rámci nově vzniklých kolonií. Osobní jména, která se na nápisech
vyskytovala, byla řeckého původu ve 100 \letterpercent{} případů:
dvougenerační identifikace pomocí osobního jména a jména rodiče
poukazuje pouze na řecký původ osobních jmen. Ze 6. st př. n. l. tak
nemáme žádné důkazy o prolínání řeckého a thráckého obyvatelstva. V
nápisech nenalézáme žádné další vyjádření identity, ani se zde
nevyskytují hledané společensko-kulturní termíny, až na jednu výjimku
výskytu formulí typických pro náhrobní nápisy v řeckém světě, jako např.
mnéma~pro označení hrobu, či náhrobního kamene samotného. Míra
tradicionalismu formy i obsahu poukazuje na relativní uzavřenost
tehdejších komunit a jistou konzervativnost projevů epigraficky aktivní
společnosti, tedy lidí, kteří se podíleli na publikování nápisů. Žádný z
dochovaných nápisů z 6. st. př. n. l. nepoukazuje na snahu předat zprávu
mimo danou komunitu a nemáme ani důkazy o vzájemné prostupnosti
jednotlivých komunit.

\subsubsection[h.ot4x6mgxo51]{Shrnutí}

Bohužel dochovaný vzorek je velmi omezený a na jeho základě není možné
hodnotit případné kontakty řecké a thrácké kultury. Dochované nápisy
pouze poukazují na fakt, že první nápisy se objevily v řecké komunitě v
prvních desetiletích po založení kolonií, a jejich charakter se velmi
podobal nápisům z jiných částí řecky mluvícího světa jak podobou, tak
obsahem. Malý počet dochovaných nápisů může poukazovat na a)
nedostatečně prozkoumané kulturní vrstvy 6. st. př. n. l., b) nejasný
charakter nápisů, který neumožňuje nápisy datovat právě do 6. st. př. n.
l., a nápisy z 6. st. př. n. l. se tak typologicky překrývají s nápisy
následujících století, c) na fakt, že společnost řeckých měst z území
Thrákie v 6. st. př. n. l. neprodukovala velké množství nápisů, protože
se potýkala jednak s interními, tak s externími problémy souvisejícími s
osídlováním již obydleného území.

\section[h.n5mo4fvc0v27]{5. st. př. n. l.}

\subsection[h.kn26qksqzvak]{Nápisy datované s přesností na jedno
století \high{\goto{{[}7{]}}[ftnt7]}}

V 5. st. př. n. l. dochází k nárůstu epigrafické produkce a rozšíření
nápisů do většího počtu řeckých komunit na pobřeží. Převládající bylo i
nadále soukromé využití nápisů, a to zejména ve funerálním funkci. V
thráckém vnitrozemí se poprvé objevují nápisy použité v rámci thráckého
funerálního kontextu, patrně patřící thrácké aristokracii a související
s prominentní pozicí, kterou aristokraté ve společnosti zastávali.

\thinrule

\goto{{[}1{]}}[ftnt_ref1]~Konkrétní způsoby relativního datování jsou
podrobně vysvětleny v Apendixu 1, věnovanému metodologii práce a
technickému řešení.

\goto{{[}2{]}}[ftnt_ref2]~Dva potenciálně nejstarší nápisy z území
Thrákie pochází z řeckého města Byzantion, které bylo založeno jako
kolonie řecké Megary v roce 667 př. n. l. Dochované nápisy (IK Byzantion
42; 53a; Vyhledání v Heuristu: IDs:14345,14358) je možné datovat pouze
velice široce: zhruba od poloviny 7. st. př. n. l. až do doby klasické,
respektive hellénistické. Jejich datace je natolik široká, že je nelze
přiřadit k jednomu či dvěma konkrétním stoletím. Protože se však může
jednat o první řecky psané epigrafické památky nalezené na území
Thrákie, považuji nutné je zmínit (ač nespadají do vybraných kategorií s
koeficientem 1 či 0,5). Typologicky se jednalo o označení váhové míry, a
označení délky, v jednom případě s uvedenou příslušností ke komunitě
obyvatel Byzantia. Nejpravděpodobnější interpretace poukazuje na místně
používané standardy, nápisy na předmětech udávající jednotnou délkovou a
váhovou míru používanou v Byzantiu a při obchodování právě s Byzantiem.
Pokud akceptujeme jejich ranou dataci, pak by se mohlo jednat o jeden z
prvních projevů politické a ekonomické autority na území Thrákie, a to v
rámci řecké komunity. Bohužel tyto dva nápisy jsou tak ojedinělé, a
nejistě datované, že mohou stejně dobře pocházet z následujících
století, a tudíž jejich výpovědní hodnota je velmi nízká.

\goto{{[}3{]}}[ftnt_ref3]~Zkopírováním následujícího výběru
identifikačních čísel nápisů do vyhledávací políčka v Heuristu dostanete
komplexní diskutovaný soubor nápisů. Vyhledání v Heuristu:
IDs:2367,8357,12320

\goto{{[}4{]}}[ftnt_ref4]~Výsledné číslo součtu nápisů nesoucích
dekoraci může být vyšší než celkový počet nápisů, protože některé nápisy
nesou více druhů dekorace, případně kombinace motivů.

\goto{{[}5{]}}[ftnt_ref5]~Vyhledání v Heuristu: IDs:8360,12321,12322

\goto{{[}6{]}}[ftnt_ref6]~Např. Perinthos-Herakleia 70: Ἡγησιπόλης εἰμὶ
τῆς Ἡγεκράτεος, \quotation{Náležím Hégésipolé, dceři Hégékratea}

\goto{{[}7{]}}[ftnt_ref7]~Vyhledání v Heuristu:
IDs:2340,12994,12996,16456,6306,6341,2168,2338,2342,2363,2368,2369,2370,2372,2398,2498,2496,2497,2500,2503,2504,2505,2508,2509,2510,2511,2512,2514,2515,2518,2521,2522,2523,2573,2575,3083,3192,3195,3196,3198,3200,3201,3203,3204,3205,8400,8831,8835,11782,15727,15729,15817,15831,15957,16182,16636,16653,16655,16656,16660

\goto{{[}8{]}}[ftnt_ref8]~Jeden nápis byl nalezen mimo region známých
měst, a není ho možné zařadit do regionu měst na pobřeží, ani ve
vnitrozemí.




\stoptext
