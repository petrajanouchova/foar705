\usepath[{/home/petra/context/}]
\usemodule[bib]
\usemodule[bibltx]
\environment env_dis



\setuppublications[alternative=apa,numbering=no,refcommand=authoryear] 
\setupbibtex[database={bibliotest},sort=author]


\startcomponent testbiblio

\startchapter[title={Test bibliografie v BibTex formatu}, marking={Test bibliografie v BibTex formatu}, reference={testbiblio}]

\section[uvod]{Úvod}

V této kapitole se věnuji podrobné charakteristice nápisů a epigrafické produkci v Thrákii obecně v rámci jednotlivých staletí. Zaměřuji především na celkovou charakteristiku epigrafické produkce, s důrazem na popis a šíření projevů mezikulturních kontaktů a vzájemné ovlivňování epigraficky aktivních komunit. Dále sleduji trendy v publikaci nápisů, rozšíření epigrafického zvyku v prostředí urbánních a venkovských komunit. Nadále mě zajímá složení těchto komunit, společenská hierarchie a šíření kulturních prvků typických pro řecky mluvící komunity v rámci osídlení thráckého pobřeží i thráckého vnitrozemí. \cite[alternative=authoryear,extras={, 123}][Tsetskhladze2004]

V této kapitole se věnuji podrobné charakteristice nápisů a epigrafické produkci v Thrákii obecně v rámci jednotlivých staletí. Zaměřuji především na celkovou charakteristiku epigrafické produkce, s důrazem na popis a šíření projevů mezikulturních kontaktů a vzájemné ovlivňování epigraficky aktivních komunit. Dále sleduji trendy v publikaci nápisů, rozšíření epigrafického zvyku v prostředí urbánních a venkovských komunit. Nadále mě zajímá složení těchto komunit, společenská hierarchie a šíření kulturních prvků typických pro řecky mluvící komunity v rámci osídlení thráckého pobřeží i thráckého vnitrozemí. \cite[Malkin1998]

V této kapitole se věnuji podrobné charakteristice nápisů a epigrafické produkci v Thrákii obecně v rámci jednotlivých staletí. Zaměřuji především na celkovou charakteristiku epigrafické produkce, s důrazem na popis a šíření projevů mezikulturních kontaktů a vzájemné ovlivňování epigraficky aktivních komunit. Dále sleduji trendy v publikaci nápisů, rozšíření epigrafického zvyku v prostředí urbánních a venkovských komunit. Nadále mě zajímá složení těchto komunit, společenská hierarchie a šíření kulturních prvků typických pro řecky mluvící komunity v rámci osídlení thráckého pobřeží i thráckého vnitrozemí. (\cite[Dietler1998])


\completepublications[criterium=text]



\stopchapter 
\stopcomponent
