\usepath[{/home/petra/context/}]
\usemodule[bib]
%\usemodule[biblatex]

\environment env_dis


\usebtxdataset[bibliotest.bib]
\usebtxdefinitions[apa]
\setupbtxrendering[sorttype=authoryear,numbering=no,criterum=text,sort]
\setupbtx[apa][alternative=authoryear,separator:names:4={\btxcomma},etaldisplay={3},separator:names:2={\btxcomma},separator:names:3={\btxcomma},separator:names:4={\btxcomma},authorconversion=invertedshort,interaction=start,separator:firstnames={\btxspace}]


\startcomponent testbiblio

\startchapter[title={Test bibliografie v BibTex formatu}, marking={Test bibliografie v BibTex formatu}, reference={testbiblio}]

\startbodymatter
\section[uvod]{Úvod}

V této kapitole se věnuji podrobné charakteristice nápisů a epigrafické produkci v Thrákii obecně v rámci jednotlivých staletí. Zaměřuji především na celkovou charakteristiku epigrafické produkce, s důrazem na popis a šíření projevů mezikulturních kontaktů a vzájemné ovlivňování epigraficky aktivních komunit. Dále sleduji trendy v publikaci nápisů, rozšíření epigrafického zvyku v prostředí urbánních a venkovských komunit. Nadále mě zajímá složení těchto komunit, společenská hierarchie a šíření kulturních prvků typických pro řecky mluvící komunity v rámci osídlení thráckého pobřeží i thráckého vnitrozemí \cite[righttext={{ \nbsp 123},{ \nbsp 116-156}}][Tsetskhladze2004, Malkin1998].

V této kapitole \cite[Cicikova2011]se věnuji podrobné charakteristice nápisů \cite[Chichikova1978] a epigrafické produkci v Thrákii obecně v rámci jednotlivých staletí. Zaměřuji především na celkovou charakteristiku epigrafické produkce, s důrazem na popis a šíření projevů mezikulturních kontaktů a vzájemné ovlivňování epigraficky aktivních komunit. Dále sleduji trendy v publikaci nápisů, rozšíření epigrafického zvyku v prostředí urbánních a venkovských komunit \cite[righttext={ \nbsp 1}][Malkin2004]. Nadále mě zajímá složení těchto komunit, společenská hierarchie a šíření kulturních prvků typických pro řecky mluvící komunity v rámci osídlení thráckého pobřeží i thráckého vnitrozemí \cite[righttext={ \nbsp 566-453}][Malkin1998].

V této kapitole \cite[authoryears][Malkin1998] se věnuji podrobné charakteristice nápisů a epigrafické produkci v Thrákii obecně v rámci jednotlivých staletí. Zaměřuji především na celkovou charakteristiku epigrafické produkce, s důrazem na popis a šíření projevů mezikulturních kontaktů a vzájemné ovlivňování epigraficky aktivních komunit. Dále sleduji trendy v publikaci nápisů, rozšíření epigrafického zvyku v prostředí urbánních a venkovských komunit. Nadále mě zajímá složení těchto komunit, společenská hierarchie a šíření kulturních prvků typických pro řecky mluvící komunity v rámci osídlení thráckého pobřeží i thráckého vnitrozemí \cite[righttext={ \nbsp 12-18}][Dietler1998].
\stopbodymatter

\startbackmatter
\startchapter[title=Bibliografie]
\placelistofpublications
\stopchapter
\stopbackmatter

\stopcomponent
