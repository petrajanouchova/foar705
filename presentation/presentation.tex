\environment envpresentation

\definebar[a][color=darkblue]

\starttext

\startlines
\blank
{\bf\tfa Students and academics are expected to produce and publish several papers per year and a book every few years. Are you one of them? 

Do you struggle every time with any of the following taks?} 
\stoplines

\startitemize[joinedup, packed] \item keep the track whether all references are correct, \item have consistently labeled figures and footnotes, \item create tables of contents, \item create indices, \item format bibliographical references, \item ...(fill in based on your experience) \stopitemize
\vfill

%\startcombination[2*1]
%{\framed[width=.5\textwidth, height=.8\textheight,  align={nothyphenated,verytolerant}, frame=off]{\setupwhitespace[big]
%\startitemize[joinedup, packed] \item keep the track whether all references are correct, \item have consistently labeled figures and footnotes, \item create tables of contents, \item create indices, \item format bibliographical references, \item ...(fill in based on your experience) \stopitemize
%}}{}
%{\externalfigure[phd.jpg][height=.3\textheight]}{}
%\stopcombination%

\framed[corner=round, width=\textwidth,height=0.15\textwidth,
backgroundcolor=gray,background=color]
{If your answer is YES to both, using \Context ~may solve your problems. \Context ~is computer markup language, allowing you to format the text how exactly you want it. It does not corrupt the file when you need it the most, and moreover, it saves your time by automatizing the tedious manual work (yes, indexes, references, re-labelling...).}

\page

\vfill

This project enables the broader academic public to set up their own high-quality publication environment. The inexperienced users can use detailed manual in order to produce books, articles, posters etc. The project aims to increase digital literacy amongst scholars and raise the awareness of existing publication tools. 
\blank

{\tfa How does it work?}
\blank

\startcombination[6*1]
{\externalfigure[research.JPG][width=.20\textwidth]}{Step 1: Do your research, write it down}
{\externalfigure[arrowright.png][width=.10\textwidth]}{}
{\externalfigure[contextlogo.png][width=.15\textwidth]}{Step 2: Format the text with \Context ~markup language}
{\externalfigure[arrowright.png][width=.10\textwidth]}{}
{\externalfigure[book.png][width=.14\textwidth]}{Step 3: Produce high-quality & publication ready book}
{\externalfigure[project.png][width=.15\textwidth]}{or article \crlf or poster \crlf  or presentation \crlf ...}

\stopcombination

\page

\placefigure[right,none]{}{\externalfigure[dis.png][height=.53\textwidth, frame=on]}
{\tfa Case study: Dissertation}
\blank

I have decided to use \Context ~to typeset my disseratation \quote{\em Hellenisation of Ancient Thrace based on epigraphic evidence}, that I am submitting in December 2016 at Charles University in Prague. I am expecting some 250 pages total, therefore any tool that could help me with the final processing, streamlining and typesetting that at the same time keep the data consistent, is exactly what I need.

I have decided to abandon the joys of MS Word (after some not-so-good experiences) and I have started learning a new language. 
Because \Context ~markup is basically a new language (and quite old at the same time). By series of tags you define how the text should behave and how it should look like. Eventhough, it sounds very technical, people who have to deal with Latin or Greek on a daily basis should be fairly comfortable with markup.

\startitemize
\item {\bf Plus}: It is logical! And it does not have exceptions unlike all spoken languages. 
\item {\bf Minus}: It takes time and enthousiasm. It is still a new language and nobody has learnt Greek in a day. So be ready to get dirty. 
\stopitemize



\page
\vfill

{\tfa 7 basic steps you need to do}
\startitemize[n]

\item Install Linux (Ubuntu 14.04)
\item Install ConTeXt (using my step-by-step manual at https://github.com/petrajanouchova/foar705/manual)
\item Install other software (GitHub, Sublime Text 3, Pandoc, Floobits if you want to work in teams etc.)
\item Get the requirements (publication rules etc.)
\item Convert your aticle/book to ConTeXt (using markup language)
\item Create a PDF with final product
\item Send it to your advisor, to your colleagues, to your publisher, put it on the internet...
\stopitemize


\page
\vfill
\startitemize

\item For user guidelines, please see https://github.com/petrajanouchova/foar705/manual
\item For additional information and official guidelines see http://wiki.contextgarden.net/
\item Source code for this presentation (compiled in \ConTeXt)~is available at: https://github.com/petrajanouchova/foar705/presentation
\item If you have any questions, contact petra.janouchova@gmail.com
\stopitemize

\stoptext


